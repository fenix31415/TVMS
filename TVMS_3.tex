\documentclass{article}
\usepackage{cmap}					% поиск в PDF
\usepackage[T2A, T1]{fontenc}
\usepackage[utf8]{inputenc}
\usepackage[english, russian]{babel}
\usepackage{enumitem} %[label=\alph*)]]
%%% Работа с русским языком
\usepackage{mathtext} 				% русские буквы в формулах
\usepackage{ccfonts,eulervm,euler}
\usepackage{bbding}
\usepackage{ulem}
\usepackage{indentfirst}
\usepackage{hyperref} %гиперссылки
\frenchspacing

\usepackage{fancyhdr} %Для шапки
\usepackage{amsthm}
\usepackage{amsmath}
\usepackage{amssymb} % R, Q,.
\usepackage{mathtools}
\usepackage{multicol}

\usepackage{indentfirst}
\parindent=1cm
\setlist[itemize]{itemsep=2pt, topsep=0pt} %norm lists
\setlist[enumerate]{itemsep=2pt, topsep=0pt} %norm lists

%%% Страница
\usepackage{geometry} % Простой способ задавать поля
\geometry{top=25mm}
\geometry{bottom=20mm}
\geometry{left=20mm}
\geometry{right=20mm}


%Определение, теорема, лемма, N.B.
\newtheorem*{defin*}{Определение}
\newtheorem{defin}{Определение}
\newtheorem*{Corollary*}{Следствие}
\newtheorem{Corollary}{Следствие}
\newtheorem*{Lemma*}{Лемма}
\newtheorem{Lemma}{Лемма}
\newtheorem{Theorem}{Теорема}
\newtheorem*{Theorem*}{Теорема}
\newtheorem{NB}{N.B}
\newtheorem*{NB*}{N.B}
\newtheorem{Example}{Пример}
\newtheorem*{Example*}{Пример}

%Proof
\newenvironment{Proof}
{\par\noindent{\bf Доказательство.}} 
{\hfill$\scriptstyle\blacksquare$}

\newcommand{\eq}[1][m]{\mathop{\equiv}\limits_{#1}}

\newcommand{\smallheader}[1]{\noindent{\bf #1 }}

\newcommand{\divs}{\,\lower.4ex\vdots\,}% a делится на b

\newcommand{\rmy}[1][m]{\mathbb{R}^{#1}}%$R^m$

\newcommand{\eqdef}{\overset{def}{\underset{}{=}}}% =def

\newcommand{\shapka}[1]{\pagestyle{fancy}\fancyhead[C]{#1}\fancyfoot{}}%Шапка

\newcommand{\chast}[2]{\dfrac{\partial #1}{\partial #2}}

%TADA https://youtu.be/9Cq56iPTQ5A?t=20
\newcommand{\THEN}{\text{\href{https://youtu.be/9Cq56iPTQ5A?t=20}{Тогда }}}

\usepackage{listings}
\usepackage{xcolor}

%New colors defined below
\definecolor{codegreen}{rgb}{0,0.6,0}
\definecolor{codegray}{rgb}{0.5,0.5,0.5}
\definecolor{codepurple}{rgb}{0.58,0,0.82}
\definecolor{backcolour}{rgb}{0.95,0.95,0.92}

%Code listing style named "mystyle"
\lstdefinestyle{mystyle}{
	backgroundcolor=\color{backcolour},   commentstyle=\color{codegreen},
	keywordstyle=\color{magenta},
	numberstyle=\tiny\color{codegray},
	stringstyle=\color{codepurple},
	basicstyle=\ttfamily\footnotesize,
	breakatwhitespace=false,         
	breaklines=true,                 
	captionpos=b,                    
	keepspaces=true,                 
	numbers=left,                    
	numbersep=5pt,                  
	showspaces=false,                
	showstringspaces=false,
	showtabs=false,                  
	tabsize=2
}

\lstset{style=mystyle}

\title{Code Listing}


\begin{document}
	\shapka{Клепов Дмитрий, М3238}
	\section{Оценка объема}
	{\bf Задание.}
	
	Методом Монте-Карло оценить объем части тела $\{F(\tilde x)\leq c\}$, заключенной в $k$-мерном кубе с ребром $[0,1]$. Функция имеет вид $F(\tilde x) = f(x_1) + f(x_2) + ... + f(x_k)$. Для выбранной надежности $\gamma\geqslant 0.95$ указать асимптотическую точность оценивания и построить асимптотический доверительный интервал для истинного значения объема.  
	Используя объем выборки $n=10^4$  и  $n=10^6$ оценить скорость сходимости и показать, что доверительные интервалы пересекаются.
	
	\bigskip
	
	{\bf Вариант 9}
	\begin{itemize}
		\item $f(x) = 7^x$
		\item $k = 10$
		\item $c = 40.4$ 
	\end{itemize}
	
	{\bf Код:}
	
	\lstinputlisting[language=Octave]{test.m}
	
	{\bf Выход:}
	
	\begin{lstlisting}
	N=10000
	S=0.959 (from 0.955114 to 0.962886)
	Delta=0.00388642
	
	N=1000000
	S=0.958263 (from 0.957871 to 0.958655)
	Delta=0.000391968
	\end{lstlisting}
	
	{\bf Вывод:}
	
	Доверительный интервал при $n = 10^6$ содержится в интервале при $n = 10^4$.\\
	При увеличении числа итераций в $100$ раз ширина доверительного интервала уменьшилось в $~10$ раз.
	
	\newpage
	
	\section{Оценка интегралов}
	{\bf Задание.}
	Построить оценку интегралов (представить интеграл как математическое ожидание функции,
	зависящей от случайной величины с известной плотностью) и для выбранной надежности $\gamma \geqslant 0.95$ указать асимптотическую точность оценки и построить асимптотический доверительный интервал для истинного значения интеграла. 
	\begin{enumerate}
		\item ${\displaystyle \int\limits_0^{\infty}\dfrac{x+1}{x+2} \cdot exp(-3x) dx}$\\
		
		{\bf Код:}\\
		\lstinputlisting[language=Octave]{test1.m}
		{\bf Выход:}
		
		\begin{lstlisting}
		Sample=0.188066
		
		N = 10000
		Value=0.187873 (from 0.187538 to 0.188207)
		Delta=0.000334562
		
		N = 1000000
		Value=0.188058 (from 0.188024 to 0.188092)
		Delta=3.3727e-005
		\end{lstlisting}
		\item ${\displaystyle \int\limits_{-3}^{4} \dfrac{\cos{x}}{x+5} dx}$
		
		{\bf Код:}
		
		\lstinputlisting[language=Octave]{file1.m}
		
		{\bf Выход:}
		
		\begin{lstlisting}
		Sample=-0.161462
		
		N = 10000
		Value=-0.16733 (from -0.190966 to -0.143695)
		Delta=0.0236355
		
		N = 1000000
		Value=-0.160921 (from -0.163237 to -0.158606)
		Delta=0.00231553
		\end{lstlisting}
	\end{enumerate}

	\bigskip

	{\bf Вывод:}
	
	Во всех случаях значение интеграла содержится в доверительных интервалах.
	При увеличении числа итераций в $100$ раз, ширина доверительного интервала уменьшилась в $10$ раз.
\end{document}