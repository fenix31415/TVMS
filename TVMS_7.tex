\documentclass{article}
\usepackage{cmap}					% поиск в PDF
\usepackage[T2A, T1]{fontenc}
\usepackage[utf8]{inputenc}
\usepackage[english, russian]{babel}
\usepackage{enumitem} %[label=\alph*)]]
%%% Работа с русским языком
\usepackage{mathtext} 				% русские буквы в формулах
\usepackage{ccfonts,eulervm,euler}
\usepackage{bbding}
\usepackage{ulem}
\usepackage{indentfirst}
\usepackage{hyperref} %гиперссылки
\frenchspacing

\usepackage{fancyhdr} %Для шапки
\usepackage{amsthm}
\usepackage{amsmath}
\usepackage{amssymb} % R, Q,.
\usepackage{mathtools}
\usepackage{multicol}

\usepackage{indentfirst}
\parindent=1cm
\setlist[itemize]{itemsep=2pt, topsep=0pt} %norm lists
\setlist[enumerate]{itemsep=2pt, topsep=0pt} %norm lists

%%% Страница
\usepackage{geometry} % Простой способ задавать поля
\geometry{top=25mm}
\geometry{bottom=20mm}
\geometry{left=20mm}
\geometry{right=20mm}


%Определение, теорема, лемма, N.B.
\newtheorem*{defin*}{Определение}
\newtheorem{defin}{Определение}
\newtheorem*{Corollary*}{Следствие}
\newtheorem{Corollary}{Следствие}
\newtheorem*{Lemma*}{Лемма}
\newtheorem{Lemma}{Лемма}
\newtheorem{Theorem}{Теорема}
\newtheorem*{Theorem*}{Теорема}
\newtheorem{NB}{N.B}
\newtheorem*{NB*}{N.B}
\newtheorem{Example}{Пример}
\newtheorem*{Example*}{Пример}

%Proof
\newenvironment{Proof}
{\par\noindent{\bf Доказательство.}} 
{\hfill$\scriptstyle\blacksquare$}

\newcommand{\eq}[1][m]{\mathop{\equiv}\limits_{#1}}

\newcommand{\smallheader}[1]{\noindent{\bf #1 }}

\newcommand{\divs}{\,\lower.4ex\vdots\,}% a делится на b

\newcommand{\rmy}[1][m]{\mathbb{R}^{#1}}%$R^m$

\newcommand{\eqdef}{\overset{def}{\underset{}{=}}}% =def

\newcommand{\shapka}[1]{\pagestyle{fancy}\fancyhead[C]{#1}\fancyfoot{}}%Шапка

\newcommand{\chast}[2]{\dfrac{\partial #1}{\partial #2}}

%TADA https://youtu.be/9Cq56iPTQ5A?t=20
\newcommand{\THEN}{\text{\href{https://youtu.be/9Cq56iPTQ5A?t=20}{Тогда }}}

\usepackage{listings}
\usepackage{xcolor}

%New colors defined below
\definecolor{codegreen}{rgb}{0,0.6,0}
\definecolor{codegray}{rgb}{0.5,0.5,0.5}
\definecolor{codepurple}{rgb}{0.58,0,0.82}
\definecolor{backcolour}{rgb}{0.95,0.95,0.92}

%Code listing style named "mystyle"
\lstdefinestyle{mystyle}{
	backgroundcolor=\color{backcolour},   commentstyle=\color{codegreen},
	keywordstyle=\color{magenta},
	numberstyle=\tiny\color{codegray},
	stringstyle=\color{codepurple},
	basicstyle=\ttfamily\footnotesize,
	breakatwhitespace=false,         
	breaklines=true,                 
	captionpos=b,                    
	keepspaces=true,                 
	numbers=left,                    
	numbersep=5pt,                  
	showspaces=false,                
	showstringspaces=false,
	showtabs=false,                  
	tabsize=2
}

\lstset{style=mystyle}

\title{Code Listing}


\begin{document}
	\shapka{Клепов Дмитрий (M3238)\\
		Обратная связь: dimkakirov43@mail.ru}
	{\bf Вариант 9}
	
	Для случайной величины $X\thicksim\exp(u)$, гипозезы $H_0\colon u=u_0$, альтернативы $H_1\colon u>u0$ построить доверительный интервал для $\alpha$ и проверить гипотезу для $\gamma$.
	
	\begin{itemize}
		\item $n=50$
		\item $u_0=35$
		\item $\alpha=0.05$
		\item $\gamma=0.9$
		\item $\overline{X}_n=40$
	\end{itemize}
	
	\bigskip
	
	{\bf ОМП и информация Фишера}
	\begin{itemize}
		\item $I(u)=1/u^2$
		\item $t_{1-\alpha}=t_{0.95}=1.96$
		\item $\hat{u}_n=\overline{X}_n=40$
		\item $I(\hat{u}_n)=1/1600$
	\end{itemize}

\bigskip

	\textbf{Доверительный интервал}:
	\begin{itemize}
		\item $\delta_n=\dfrac{t_{1-\alpha}}{\sqrt{n\cdot I(\hat{u}_n)}}=11.087$
		\item $\left[\overline{X}_n-\delta_n;\overline{X}_n+\delta_n\right]=[28.913;51.087]$
	\end{itemize}

	{\bf  Вывод:}
	$u_0=35\in I_n$, то есть гипотеза принимается.
	
	\bigskip
	
	\textbf{Правосторонняя альтернатива}
	\begin{itemize}
		\item $c_\gamma=1.29$
		\item $\theta_0=u_0=35$
	\end{itemize}

	$\Psi_{n,\alpha}^{*}=
	\left\{\begin{array}{l l}
	1, & \sqrt{n\cdot I(\theta_0)}\cdot(\hat{\theta}_n-\theta_0)\geqslant c_\gamma\\
	0, & \text{otherwise}
	\end{array}\right. $
	
	{\bf  Вывод:} $\sqrt{n\cdot I(\theta_0)}\cdot(\hat{\theta}_n-\theta_0)\approx1.01$, то есть гипотеза принимается.
	
	
\end{document}
